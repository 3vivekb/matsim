\documentclass[ngerman]{beamer}

\mode<presentation>
{
%themen ohne navigation
%\usetheme{default}
%\usetheme{Bergen} %naja sagen wir innovativ nordisch
%\usetheme{Boadilla} % komisch
%\usetheme{Madrid} %ganz schick aber mit viel info auf folie
%\usetheme{AnnArbor}% grasse farben
%\usetheme{CambridgeUS}
%\usetheme{Pittsburgh} % ganz gut aber folien�berschrift rechts
%\usetheme{Rochester} %schlicht aber ok
%themen mit baum navigation
%\usetheme{Antibes}
%\usetheme{JuanLesPins} %ganz gut
%\usetheme{Montpellier} %ganz ok
%themen mit inhaltsverzeichnis
%\usetheme{Berkley}
%\usetheme{PaloAlto} % schick bis auf die kapitelnummern
%\usetheme{Goettingen} %ganz gut
%\usetheme{Marburg} %h�sslich
%\usetheme{Hannover}
%themen mit mini frame navigation
%\usetheme{Berlin}
%\usetheme{Ilmenau}
%\usetheme{Dresden}
%\usetheme{Darmstadt}
%\usetheme{Frankfurth}
%\usetheme{Singapore}
%\usetheme{Szeged}
%themen mit abschnitt /unterabschnitt gliederung
%\usetheme{Copenhagen}
%\usetheme{Luebeck}
%\usetheme{Malmoe}
%\usetheme{Warsaw}

%titelseite aufz�hlungen block theorem und beweisumgebungen bilder tabellen etc bestimmen.
%\useinnertheme{default}
\useinnertheme{circles}
%\useinnertheme{rectangles}
%\useinnertheme{rounded}
%\useinnertheme{inmargin}


%kopf- fu�zeile, sidebars, logo, folientitel
%\useoutertheme{default}
%\useoutertheme{infolines}
%\useoutertheme{miniframe} %geht nicht
%\useoutertheme{smoothbars} %�bersicht in der kopfzeile
%\useoutertheme{sidebar}
%\useoutertheme{split}
%\useoutertheme{shadow} %f�r schattierungen
%\useoutertheme{tree} %baum�bersicht hinzuf�gen
%\useoutertheme{smoothtree}%auskommentiert

%farbschemata
%vollst�ndige
\usecolortheme{default}
%\usecolortheme{albatross} %zu blau
%\usecolortheme{beetle} %sehr dunkel
%\usecolortheme{crane} %halt gelb
%\usecolortheme{dove} %halt wei�
%\usecolortheme{fly} %grau
%\usecolortheme{seagull} %grau blau schlicht
%\usecolortheme{wolverine} %kohl regiert die republik
%\usecolortheme{beaver}

%�u�ere
%\usecolortheme{seahorse} %auskommentiert
%\usecolortheme{dolphin}
%\usecolortheme{whale} %nee

%innere
%\usecolortheme{lily}
%\usecolortheme{orchid}
%\usecolortheme{rose}

%\usefonttheme{professionalfonts}
%\usefonttheme{serif}
\usefonttheme{structurebold}
%\usefonttheme{stuctureitalicserif}
%\usefonttheme{structuresmallcapserif}

%\setbeamercovered{transparent}
}

% F�r Systeme die Unicode benutzen, ansonsten [latin1]
\usepackage[utf8]{inputenc}
\usepackage[T1]{fontenc}



% deutsche Silbentrennung
\usepackage[ngerman]{babel}
%neue deutsche silbentrennung
\usepackage{ngerman}

%graphiken
\usepackage{graphicx}

%mathe schriften
\usepackage{amsfonts}

%bessere schriftdarstellung in pdfs
%\usepackage{ae}

% font definitions, try \usepackage{ae} instead of the following
% three lines if you don't like this look
%\usepackage{mathptmx}
%\usepackage[scaled=.90]{helvet}
%\usepackage{courier}

\usepackage{ae,aecompl}
