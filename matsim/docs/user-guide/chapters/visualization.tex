There are two visualizers available for MATSim. The original, open source 
visualizer is \href{http://matsim.org/docs/extensions/otfvis}{OTFVis}, which is
a MATSim extension. It requires current OpenGL drivers. The source code is 
available, so you can add your own information visualization code. On the 
other hand, there is currently little support for it from our part.

Then there is Via, a commercial visualizer developed by Senozon. It  has more
features, a better UI, and it is more stable. On the other hand, it visualizes
output files from simulation runs, whereas OTFVis runs in the same VM as
MATSim and can peek into the running simulation.

The supported way of programming your own data analysis or visualization code
is to analyze MATSim output in the form of Events, either reading in the 
events.xml file, or writing an EventHandler and receiving Events programmatically.

\section{Senozon Via}

Via can be obtained from the \href{http://senozon.com/products/via}{Senozon website}.
While the application is commercial, a limited version is available freely from the website. 
A brief introduction to using Senozon Via is part of the "Learning MATSim in 8 Lessons"
tutorial in the \href{http://www.matsim.org/docs/tutorials/8lessons/getting-started}{Getting Started} lesson.

Via is able to visualize most of MATSim's data (network, agent plans, transit schedule, facilities, counts)
along additional data (e.g. shape files, GPS traces).
It allows to analyze and visualize the outcome of MATSim simulations by loading the generated events-file.


On the relation between Senozon (a private company), Senzon Via (a commercial software), 
MATSim (an open source project/software) and MATSim OTFVis (an open source software):
\begin{itemize}\styleItemize
	\item Historically, MATSim is open source. An important reason for this was that multiple teams contribute, and we wanted to make progress rather than sorting out the intellectual property.
	\item However, this community is unable to provide support for any and all requests that may come up. As a result, the commercial company \href{http://www.senzon.com/}{Senozon} was founded by two long-time MATSim developers, which provides commercial support for such situations.
	\item Senozon also helps significantly with the development and maintenance of the MATSim core. The open source community and Senozon have a shared interest in a functional and robust MATSim core: Both our academic research and Senozon's commercial success depend on this.
	\item In addition, Senozon has developed the \href{http://senozon.com/products/via}{MATSim visualization and analysis software Via}.  OTFVis remains available but maintenance is limited. In  particular, please understand that we are unable to provide support for specific hardware configurations or specific query requests.
\end{itemize}

\section{Events analysis}

In  order to write MATSim events handlers, some amount of Java programming  is necessary. Material can thus found in the api-users section of the documentation, see \href{http://www.matsim.org/node/17}{here}.
