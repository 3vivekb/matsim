\NextFile{BuildingScenarios.html}
\chapter{Building New Scenarios}
\label{sec:BuildingScenarios}

\authorsOfDoc{Marcel Rieser}
 
\bigskip

\begin{chapter-intro}
Starting a new scenario (our term for the application of MATSim to a 
region/area) can appear quite cumbersome at the first glace, as a lot of data
preparation may be required. This chapter gives first an overview of the input
data typically required for running a MATSim scenario, and then gives examples
how such data is generated for existing scenarios.
\end{chapter-intro}

\section{Typical Input Data Sets}
MATSim uses multiple files to store the different types of data it uses for the
simulation. Tab.~\ref{tab:BuildingScenarios:InputDataSets} gives an
overview over files you may typically encounter when working with MATSim.

Not all files are always required. Very simple simulations can be run
only with a configuration file and the description of the network and
the population containing the agents' plans. For additional functionality, e.g.
for the simulation of public transport, additional files might be required.

\begin{table}[htp]
\begin{tabular}{ll}
\hline
{\tt config.xml}          & configuration options for MATSim \\
{\tt network.xml}         & description of the (road) network \\
{\tt population.xml}      & the travel demand, i.e. the list of agents
and their day plans
\\
{\tt facilities.xml}      & information about locations where
activities can be performed\\
{\tt transitSchedule.xml} & information about transit stop locations
and transit services\\
{\tt transitVehicles.xml} & description of the vehicles used for
public transport services\\
{\tt counts.xml}          & hourly volumes from real-world counting
stations for comparison\\
\hline
\end{tabular}
\caption{Files often used with MATSim}
\label{tab:BuildingScenarios:InputDataSets}
\end{table}

In the following, small examples of these files will be shown and the data they
contain discussed.

\begin{note}
Some of the files, especially {\tt population.xml}, but also {\tt network.xml} 
or {\tt facilities.xml}, might get quite large. To save space, MATSim supports
reading and writing the data in a compressed format. MATSim uses  
GZIP-compression for this. Thus, in many cases, the file names have the 
additional suffix {\tt .gz}, as in {\tt population.xml.gz}. MATSim automatically
detects if files are compressed or should be written compressed based on the 
filename.
\end{note}

\subsection{Configuration}

\begin{xml-file}[caption=An example of a config.xml,
label=lst:BuildingScenarios:configXml]
<?xml version="1.0" ?>
<!DOCTYPE config SYSTEM "http://www.matsim.org/files/dtd/config_v1.dtd">
<config>

	<module name="network">
		<param name="inputNetworkFile" value="example/network.xml" />
	</module>

	<module name="plans">
		<param name="inputPlansFile" value="example/population.xml.gz" />
	</module>

	<module name="controler">
		<param name="outputDirectory" value="./output/" />
		<param name="firstIteration" value="0" />
		<param name="lastIteration" value="10" />
	</module>
	
</config>
\end{xml-file}

The configuration file, often just referred to as \emph{config file}
or as \emph{config.xml}, builds the connection between the user and MATSim.
It contains a list of settings which influence how the simulation behaves.

All configuration parameters are simple pairs of a parameter name and a
parameter value. The parameters are grouped into logical groups. For example,
there is a group with settings related to the Controler like the number of
iterations, or there is another group with settings related to the simulation,
e.g. the end time of the simulation.
Listing~\ref{lst:BuildingScenarios:configXml} shows a very short example of a
configuration file which specifies the network and travel demand data to be used
along with some settings for the Controler.

The list of available parameters and valid parameter values may vary from
release to release. Although we try to keep this stable, due to changes in the
software, most notably by new features, settings may change. To get a list of
all available settings currently available, run the following command:
\begin{lstlisting}
java -cp matsim.jar org.matsim.run.CreateFullConfig fullConfig.xml
\end{lstlisting}
This command will create a new config file {\tt fullConfig.xml} which contains
the full list of available parameters along with their default values. This
makes it easy to see what settings are available. To use and modify certain
settings, the lines with the corresponding parameters can be copied to the
config file specific for the scenario to be simulated and the parameter values
be modified in that file.


\subsection{Network}

\begin{xml-file}[caption=An example of a network.xml,
label=lst:BuildingScenarios:networkXml]
<?xml version="1.0" encoding="utf-8"?>
<!DOCTYPE network SYSTEM "http://www.matsim.org/files/dtd/network_v1.dtd">

<network name="example network">
	<nodes>
		<node id="1" x="0.0" y="0.0"/>
		<node id="2" x="1000.0" y="0.0"/>
		<node id="3" x="1000.0" y="1000.0"/>
	</nodes>
	<links>
		<link id="1" from="1" to="2" length="3000.00" capacity="3600" 
		                           freespeed="27.78" permlanes="2" modes="car" />
		<link id="2" from="2" to="3" length="4000.00" capacity="1800" 
		                           freespeed="27.78" permlanes="1" modes="car" />
		<link id="3" from="3" to="2" length="4000.00" capacity="1800" 
		                           freespeed="27.78" permlanes="1" modes="car" />
		<link id="4" from="3" to="1" length="6000.00" capacity="3600" 
		                           freespeed="27.78" permlanes="2" modes="car" />
	</links>
</network>
\end{xml-file}

The network describes the infrastucture on which the agents (or the vehicles,
respectively), can move around. The network consists of \emph{nodes} and
\emph{links} (in graph theory, these are typically called \emph{vertices} and
\emph{edges}). Listing~\ref{lst:BuildingScenarios:networkXml} shows an example
of a simple description of a network in MATSim's XML data format.

Each element has an identifier \emph{id}. Nodes are described by an X and a Y
coordinate value. Links have more attributes: The \emph{from} and \emph{to}
attribute reference nodes and describe the geometry of the network. Additional
attributes describe the traffic-related aspects of the network:
\begin{itemize}\styleItemize
  \item the \emph{length} of the link, typically in meters (see
  Sec.~\ref{sec:BuildingScenarios:Units}).
  \item the \emph{flow capacity} of the link, i.e. the number of vehicles that
  can pass the link, typically in vehicles per hour.
  \item the \emph{freespeed} is the maximum speed at which vehicles are allowed
  to travel along the link, typically in meters per seconds.
  \item the \emph{number of lanes} ({\tt} permlanes) available in the direction
  specified by the {\tt from} and {\tt to} nodes.
  \item the list of \emph{modes} allowed on the link. This is a comma-separated
  list, e.g. {\tt modes="car,bike,taxi"}.
\end{itemize}
Note that all links are uni-directional. If a road can be travelled in both
directions, two links have to be defined with alternating {\tt to} and {\tt
from} attributes (see links with id {\tt 2} and {\tt 3} in the example given in
Listing~\ref{lst:BuildingScenarios:networkXml}).
Thus, the network can be seen as a directed graph.

\subsection{Demand}

The travel demand for MATSim is described by the agents' day plans. The full set
of agents is typically the \emph{population}, hence the filename {\tt
population.xml}. Alternatively, {\tt plans.xml} is also commonly used in MATSim,
as the population file essentially contains a list of day plans.

The population contains the data in a hierarchical structure, as shown in
Listing~\ref{lst:BuildingScenarios:populationXml}:
\begin{itemize}\styleItemize
  \item The population contains a list of persons.
  \item Each person contains a list of plans.
  \item Each plan contains a list of \emph{Activities} and \emph{Legs}.
\end{itemize}
Exactly one plan per person is marked as \emph{selected}. The selected plan of
each agent is the plan that gets executed by the mobility simulation. During the
replanning stage, a different plan might get marked as being selected. A plan
can contain a score as attribute. The score gets calculated and stored in the
plan during the scoring stage, after the plan was executed by the mobility
simulation.

The list of activities and legs in each plan describe the planned actions by
each agent. Activities have a type assigned and have---except for the last
activity in a day plan---an end time defined (There are some exceptions where
activities have a duration instead of an end time. Such activities are often
automatically generated by routing algorithms and are thus not described in
this guide). To describe the location where an activity takes place, the
activity is either assigned a coordinate by giving an x and y attribute value,
or has a link assigned which describes from which link the activity can be
reached. As the simulation requires the link attribute, the Controler calculates
the nearest link for a given coordinate in the case the attribute is missing and
only an x and y coordinate value is given or any activity.

\begin{xml-file}[caption=An example of a population.xml,
label=lst:BuildingScenarios:populationXml]
<?xml version="1.0" encoding="utf-8"?>
<!DOCTYPE population SYSTEM "http://www.matsim.org/files/dtd/population_v5.dtd">
<population>
	<person id="1">
		<plan selected="yes" score="93.2987721">
			<act type="home" link="1" end_time="07:16:23" />
			<leg mode="car">
				<route type="links">1 2 3</route>
			</leg>
			<act type="work" link="3" end_time="17:38:34" />
			<leg mode="car">
				<route type="links">3 1</route>
			</leg>
			<act type="home" link="1" />
		</plan>
	</person>
	<person id="2">
		<plan selected="yes" score="144.39002">
			\ldots
		</plan>
	</person>
</population>
\end{xml-file}

Legs describe how agents plan to travel from one location to the next one. Each
leg must have a transport mode assigned. Optionally, legs may have an
attribute {\tt trav\_time} which describes the expected travel time for this
leg. For a leg to be simulated, it must contain a route. The format of a
route depends on the mode of a leg. For car-legs, the route lists the links that
the agent has to travel along in the given order, while for transit-legs
information about the stop locations and expected transit services are stored.

An agent starts a leg directly after the previous activity (or leg) has ended.
Depending on the mode, the mobility simulation might handle the agent
differently. By default, car- and transit-legs are well-supported by the
mobility simulation. If the mobsim encounters a mode it does not know, it
defaults to \emph{teleportation}: In this case, the agent is removed from the
simulated reality, and after the leg's expected travel time has passed,
re-inserted at the agent's target location.

\bigskip

The population data format is one of the most central data structures in
MATSim and might be a bit overwhelming at first. Luckily, to get started, only a
small subset must be known of it.
Listing~\ref{lst:BuildingScenarios:minimalPopulationXml} shows how a minimal
population file could look like. Most notably, the following simplications can
be made:
\begin{itemize}\styleItemize
  \item Each person needs exactly one plan.
  \item The plan does not need to be selected or have a score.
  \item Activities can be located just by their coordinates.
  \item Activities should have a somewhat meaningful end-time.
  \item Legs only need a mode, but no routes.
\end{itemize}
When a simulation is started, MATSim's Controler will load such a file and then
automatically assign the nearest linnk to each activity and calculate a suitable
route for each leg. This makes it easy to get started quickly.

\begin{xml-file}[caption=Minimal population.xml required to start MATSim,
label=lst:BuildingScenarios:minimalPopulationXml]
<?xml version="1.0" encoding="utf-8"?>
<!DOCTYPE population SYSTEM "http://www.matsim.org/files/dtd/population_v5.dtd">
<population>
	<person id="1">
		<plan>
			<act type="home" x="5.0" y="8.0" end_time="08:00:00" />
			<leg mode="car">
			</leg>
			<act type="work" x="1500.0" y="890.0" end_time="17:30:00" />
			<leg mode="car">
			</leg>
			<act type="home" x="5.0" y="8.0" />
		</plan>
	</person>
	<person id="2">
		\ldots
	</person>
</population>
\end{xml-file}

\subsection{Public Transport}
\begin{xml-file}[caption=An example of a schedule.xml]
<?xml version="1.0" encoding="UTF-8"?>
<!DOCTYPE transitSchedule SYSTEM "http://www.matsim.org/files/dtd/transitSchedule_v1.dtd">
<transitSchedule>
	<transitStops>
		<stopFacility id="1" x="990.0"  y="0.0"   name="Adorf"    linkRefId="1" isBlocking="false"/>
		<stopFacility id="2" x="1100.0" y="980.0" name="Beweiler" linkRefId="2" isBlocking="true"/>
		<stopFacility id="3" x="0.0"    y="10.0"  name="Cestadt"  linkRefId="3" isBlocking="false"/>
	</transitStops>
	<transitLine id="Blue Line">
		<transitRoute id="1">
			<description>Just a comment.</description>
			<transportMode>bus</transportMode>
			<routeProfile>
				<stop refId="1" departureOffset="00:00:00"/>
				<stop refId="2" arrivalOffset="00:02:30" departureOffset="00:03:00" awaitDeparture="true"/>
				<stop refId="3" arrivalOffset="00:05:00" awaitDeparture="true"/>
			</routeProfile>
			<route>
				<link refId="1"/>
				<link refId="2"/>
				<link refId="3"/>
			</route>
			<departures>
				<departure id="1" departureTime="07:00:00" vehicleRefId="12"/>
				<departure id="2" departureTime="07:05:00" vehicleRefId="23"/>
				<departure id="3" departureTime="07:10:00" vehicleRefId="34"/>
			</departures>
		</transitRoute>
	</transitLine>
</transitSchedule>
\end{xml-file}

\begin{xml-file}[caption=An example of transitVehicles.xml]
<?xml version="1.0" encoding="UTF-8"?>
<vehicleDefinitions xmlns="http://www.matsim.org/files/dtd"
       xmlns:xsi="http://www.w3.org/2001/XMLSchema-instance"
       xsi:schemaLocation="http://www.matsim.org/files/dtd http://www.matsim.org/files/dtd/vehicleDefinitions_v1.0.xsd">
	<vehicleType id="1">
		<description>Small Train</description>
		<capacity>
			<seats persons="50"/>
			<standingRoom persons="30"/>
		</capacity>
		<length meter="50.0"/>
	</vehicleType>
	<vehicle id="tr_1" type="1"/>
	<vehicle id="tr_2" type="1"/>
</vehicleDefinitions>
\end{xml-file}

\todo{MR}


\subsection{Counts}
\begin{xml-file}[caption=An example of a counts.xml]
<?xml version="1.0" encoding="UTF-8"?>
<counts xmlns:xsi="http://www.w3.org/2001/XMLSchema-instance" 
        xsi:noNamespaceSchemaLocation="http://matsim.org/files/dtd/counts_v1.xsd" 
        name="test" desc="test counting stations" year="2014">
	<count loc_id="2" cs_id="005">
		<volume h="1" val="10.0"></volume>
		<volume h="2" val="1.0"></volume>
		<volume h="3" val="2.0"></volume>
		<volume h="4" val="3.0"></volume>
		<volume h="5" val="4.0"></volume>
		<volume h="6" val="5.0"></volume>
		<volume h="7" val="6.0"></volume>
		<volume h="8" val="7.0"></volume>
		<volume h="9" val="8.0"></volume>
		<volume h="10" val="9.0"></volume>
		<volume h="11" val="10.0"></volume>
		<volume h="12" val="11.0"></volume>
		<volume h="13" val="12.0"></volume>
		<volume h="14" val="13.0"></volume>
		<volume h="15" val="14.0"></volume>
		<volume h="16" val="15.0"></volume>
		<volume h="17" val="16.0"></volume>
		<volume h="18" val="17.0"></volume>
		<volume h="19" val="18.0"></volume>
		<volume h="20" val="19.0"></volume>
		<volume h="21" val="20.0"></volume>
		<volume h="22" val="21.0"></volume>
		<volume h="23" val="22.0"></volume>
		<volume h="24" val="23.0"></volume>
	</count>
</counts>
\end{xml-file}

\todo{MR}

\section{Units and Conventions Used}
\label{sec:BuildingScenarios:Units}
\todo{MR: id: any string, but please no white spaces and commas, coordinate system, mostly meter, seconds}

\section{Coordinate Systems}
\label{sec:BuildingScenarios:CoordinateSystems}

\subsection{Preparing Your Data in the Right Coordinate System}

In several input files, you need to specify coordinates, e.g. for the nodes of
the network. It is strongly suggested \emph{not} to use WGS84 coordinates (i.e.
GPS coordinates, or any other kind of spherical coordinates; coordinates ranging
from -180 to +180 in west-east direction, and from -90 to +90 in south-north
direction). MATSim needs to calculate distances between two points in several
places of the code. The calculation of distances between spheric coordinates is
very complex and potentially slow. Instead, MATSim uses the simple Pythagoras'
theorem, but this requires the coordinates to be in a Cartesian coordinate
system. Thus is is stronlgy advised to use a Cartesian coordinate system along
with MATSim, preferably one where the distance unit corresponds to one meter.

Many countries and regions have custom coordinate system defined, optimized for
usages in their apropriet areas. It might be best to ask some GIS specialists in
your region of interest what the most commonly used local coordinate system is
and use that as well for your data. 

If you don't have any clue about what coordinate system is used in your region,
it might be best to use the Universal Transverse Mercator coordinate system.
This coordinate system divides the world into multiple bands, each six degrees
width and separated into a northern and southern part, which it calls UTM zones
(see \url{http://en.wikipedia.org/wiki/UTM_zones#UTM_zone} for more details).
For each zone, an optimized coordinate system is defined. Choose the UTM zone
which covers your region (Wikipedia has a nice map showing the zones) and use
its coordinate system.

\subsection{Telling MATSim about Your Coordinate System}

In some places, MATSim requires to know which coordinate system your data is in.
You have multiple ways to specify the coordinate system you use. The easiest one
is to use the so-called ``EPSG codes''. Most of the commonly used coordinate
systems got standardized and numbered. The EPSG code uniquely identifies a
coordinate system and can be directly used by MATSim. As an alternative, MATSim
can also parse the description of a coordinate system in the so-called WKT
format. As the WKT format is much more error-prone it is suggested to use EPSG
codes whenever possible.

To find the correct EPSG code for your coordinate system (e.g. for one of the
UTM zones), the website \url{http://www.spatialreference.org} is of great use.
Search on this website for your coordinate system, e.g. for ``WGS84 / UTM Zone
8N'' (for the northern-hemisphere UTM Zone 8) to find a list of matching
coordinate systems along with their EPSG codes.


For some operations, MATSim must know the coordinate system your data is in.
Some analyses may create output to be visualized in Google Earth for example,
where the coordinates need to be converted back to WGS84. The coordinate system
used by your data can be specified in the config file:

\begin{lstlisting}{language=XML}
<module name="global">
  <param name="coordinateSystem" value="EPSG:32608" />
</module>
\end{lstlisting}

This allows MATSim to work with your coordinates and convert them whenever
needed.


old stuff from here
================

Given these two data items, you can already start building your own
scenario.The "\href{http://www.matsim.org/docs/tutorials/learningIn3days}{Learning MATSim in 3 days}"-Tutorial gives you an introduction on how to build your own scenario.

\subsubsection{Import from VISUM}

See \href{http://matsim.org/javadoc/org/matsim/visum/package-summary.html}{here for javadoc}, and \href{http://matsim.org/xref/org/matsim/visum/package-summary.html}{here for code}.

\subsubsection{Programming}

In many cases, using pre-configured software is not possible because  there are just too many possibilities of how input could look like.  Although matsim is not there yet, these should be api-only use cases,  i.e. they should only use the "stable" api. Therefore, the  following are under the api-users section of the documentation:
\begin{itemize}
	\item Additional information about network generation is \href{http://matsim.org/node/588}{here}.
	\item additional information about initial demand generation is \href{http://matsim.org/node/340}{here}.
\end{itemize}

\subsubsection{Information concerning specific scenarios}

The following sections contains various information for building a scenario that presumably goes beyond what is in the tutorial.


\section{Using MATSim for Switzerland}

\subsubsection{General remarks}

This is material that was in the tutorial "Learning MATSim in 3 days". We moved it to here, but did not check the content.

A. Horni: Shortly, there will be a working paper describing the usage of MATSim for Switzerland., 02.05.2011

\subsubsection{Some data sources}
\begin{itemize}
	\item Microcensus (BfS):\\ \href{http://www.bfs.admin.ch/bfs/portal/de/index/themen/11/07/01/02/01.html}{www.bfs.admin.ch/bfs/portal/de/index/themen/11/07/01/02/01.html}
	\item Census (BfS):\\ \href{http://www.bfs.admin.ch/bfs/portal/en/index/infothek/erhebungen__quellen/blank/blank/vz/uebersicht.html}{www.bfs.admin.ch/bfs/portal/en/index/infothek/erhebungen\_\_quellen/blank/blank/vz/uebersicht.html}
	\item ASTRA traffic counts:\\ \href{http://www.astra.admin.ch/verkehrsdaten/00299/00303/index.html?lang=en}{www.astra.admin.ch/verkehrsdaten/00299/00303/index.html}
	\item Business census (BfS):\\ \href{http://www.bfs.admin.ch/bfs/portal/en/index/infothek/erhebungen__quellen/blank/blank/bz/01.html}{www.bfs.admin.ch/bfs/portal/en/index/infothek/erhebungen\_\_quellen/blank/blank/bz/01.html}
	\item Border crossing traffic (IVT, BfS):\\ \href{http://www.bfs.admin.ch/bfs/portal/de/index/themen/11/07/04/blank/01/01.html}{www.bfs.admin.ch/bfs/portal/de/index/themen/11/07/04/blank/01/01.html}
\end{itemize}


%%\vfill\eject
%%\section{Using MATSim from Urbansim (for the PSRC region)}

%%Deprecated (and soon to be removed). See \href{http://matsim.org/extensions/matsim4urbansim}{here} instead.



%%\sout{Check  out the opus source tree from www.urbansim.org . (The source tree  is the tree containing directories such as opus\_core or opus\_gui.)}

%%\sout{In this source tree, there should be a directory opus\_matsim .}

%%\sout{There is documentation in opus\_matsim/docs .}

