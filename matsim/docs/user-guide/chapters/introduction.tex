\NextFile{Introduction.html}
\chapter{Introduction}

The "tutorial" section contains "reduced" information about how to find your way into matsim.

This "user's guide" section contains additional information,  concentrating on features and details that are not explained in the  tutorials. Clearly, there may be overlap.

\chapter{Features}

\authors{Marcel Rieser}

\bigskip

The following list shows the key features of MATSim:

\textbf{Fast Dynamic and Agent-Based Traffic Simulation}
\\  In many cases, MATSim only takes a couple of minutes for a single  simulation of a complete day of traffic. This includes the completely  time-dynamic simulation of motorized individual traffic as well as the  handling of agents using other modes of transport.

\textbf{Supports Large Scenarios}
\\  MATSim is able to simulate scenarios with several millions agents or  network with hundreds of thousands of streeets. All you need is a  current, fast desktop computer with plenty of memory. Additionally,  MATSim allows you to only simulate a certain percentage of the traffic,  speeding up the simulation even more while reducing memory consumption,  and still generate useful results.

\textbf{Sophisticated Interactive Visualizer}
\\  Forget aggregated results! MATSim provides a fast Visualizer that can  display the location of each agent in the simulation and what it is  currently doing. It can even connect to a running simulation, allowing  interactively querying agents' states, visualizing agents' routes or  perform live analyses of the network state.

\textbf{Versatile Analyses and Simulation Output}
\\  During the simulation, MATSim collects several key values from the  simulation and outputs them to give you a quick overview of the current  state of the simulation. Among other results, it can compare the  simulated traffic to real world data from counting stations, displaying  the results interactively in Google Earth. Additionally, MATSim provides  detailed output from the traffic microsimulation, which can easily be  parsed by other applications to create your own special analyses.

\textbf{Modular Approach
\\}MATSim  allows for easy replacement or addition of functionality. This allows  you to add your own algorithms for agent-behavior and plug them into  MATSim, or use your own transport simulation while using MATSim's  replanning features.

\textbf{Open Source}
\\ MATSim is  distributed under the Gnu Public License (GPL), which means that MATSim  can be downloaded and used free of charge. Additionally, you get the  complete Source Code which you may modify within certain constraints  (see the license for more details). Written in Java, MATSim runs on all  major operating systems, including Linux, Windows and Mac OS X.

\textbf{Active Development and Versatile Usage of MATSim}
\\  Researchers from several locations are currently working on MATSim.  Core development takes place at the Berlin Institute of Technology (TU  Berlin), the Swiss Federal Institute of Technology (ETH) in Zurich, as  well as in a start-up founded by two former PhD students. Additional  development (as far as we are aware of) currently takes place in South  Africa, Germany (Munich, Karlsruhe) as well as other places around the  world. This distribution of development ensures that MATSim not only  works for one scenario/context, but can be adapted to many different  scenarios.
