\authorsOfDoc{Marcel Rieser}
 
\bigskip

\begin{chapter-intro}
MATSim comes without any easy to use graphical interface. Instead, it uses text
files for configuration and uses a command line interface to start simulations.
This chapter shows you how to start MATSim simulations in a number of different
computing environments.
\end{chapter-intro}


\section{Prerequisites}

MATSim is written in Java, a programming language which allows to write
applications that run on a large variety of computers. As scenarios can become
quite large, they may consume large amounts of memory (RAM). Very large
scenarios should be run on dedicated servers with enough resources.

In general, to use MATSim, the following requirements must be met:
\begin{itemize}\styleItemize
  \item Java SE 7 must be installed. The latest version of the Java Runtime
  Environment (JRE) can be downloaded from \url{http://java.oracle.com}.
  \item Have at least 2 GB memory for running the examples. For larger
  scenarios, more memory will be required.
  \item Have enough free hard drive space. The provided examples will occupy
  only a few megabytes, but large scenarios can easily use multiple gigabytes of
  disk space.
  \item Last but not least, you need a version of MATSim.
\end{itemize}
This chapter assumes you are using a release (or nightly build) of MATSim which
comes pre-packaged as a jar-file. This chapter will not explain how to compile
and run MATSim based on its program source files. In the remainder of this
chapter, {\tt matsim.jar} will be used to refer to the jar-file of MATSim. The
actual name might differ, e.g. it might include a version number like {\tt
matsim-0.6.0.jar} or {\tt matsim-20140114.jar}.


\section{Using MATSim from a Command Line}
To run MATSim, one needs a configuration file and an estimation, how much memory
the simulation will consume (Don't fear this, you'll get used to this quite
fast). As MATSim comes without graphical user interface, it needs to be run on
the command line. In Linux or Mac OS X, this is typically done using a Terminal
application. In Windows, the Power Shell or Command Prompt can be used.

On the command line, type the following command, but substitute the correct
paths:
\begin{verbatim}
java -Xmx512m -cp /path/to/matsim.jar
      org.matsim.run.Controler /path/to/config.xml
\end{verbatim}
Note that the commands should always be written on one line, they are shown in
this tutorial on multiple lines only for readability.

As an example, on Linux this could look like:
\begin{verbatim}
java -Xmx512m -cp /home/username/matsim/matsim.jar 
      org.matsim.run.Controler /home/user/matsim/input/config.xml
\end{verbatim}

On Mac OS X, it could look like this:
\begin{verbatim}
java -Xmx512m -cp /Users/username/matsim/matsim.jar 
      org.matsim.run.Controler /Users/user/matsim/input/config.xml
\end{verbatim}

On Windows, an example command could be:
\begin{verbatim}
java -Xmx512m -cp C:\MATSim\matsim.jar 
      org.matsim.run.Controler C:\MATSim\input\config.xml
\end{verbatim}

Such a command exists of multiple parts:
\begin{itemize}
  \item {\tt java} tells the system that you want to run Java.
  \item {\tt -Xmx512m} tells Java that it should use up to 512 MB of memory.
  This is typically enough to run the small examples. For larger scenarios, you
  might need more memory: {\tt -Xmx3g} would allow Java to use up to 3 GB of
  memory.
  \item {\tt -cp /path/to/matsim.jar} tells Java where to find the MATSim code.
  \item {\tt org.matsim.run.Controler} tells Java which class (think of ``entry
  point'') it should start running. In most cases, the default MATSim Controler
  is the class you'll need to run simulations.
  \item {\tt /path/to/config.xml} tells MATSim which config file is to be used.
\end{itemize}

In the case you have relative paths in your config file, make sure to start
MATSim in the correct directory. It will interpret all relative paths based on
the directory where the Java process got started, and not where the config file
is located.

